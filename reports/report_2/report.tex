\documentclass{article}
\usepackage{graphicx}
\usepackage{titlesec}
\usepackage{float}
\usepackage{fancyhdr}
\usepackage[english]{babel}
\usepackage[margin=2.2cm]{geometry}
\usepackage[utf8]{inputenc}
\usepackage[T1]{fontenc}
\usepackage{subcaption}

\graphicspath{{figures/}}

\titleformat*{\section}{\LARGE\bfseries}
\titleformat*{\subsection}{\large\bfseries}
\titleformat*{\subsubsection}{\small\bfseries}
 
\pagestyle{fancy}
\fancyhf{}
\rhead{Joachim Tapparel and Lennard Ludwig}
\lhead{Mini-project 2 : Framework}
\cfoot{\thepage}

\begin{document}
\thispagestyle{fancy}

\section{Architecture}

Each module implements the following functions : 

\paragraph{forward(self, *input)}
The function computes the output from the given input and the parameters of the module.
    
\paragraph{backward(self, *gradwrtoutput)}
The function computes the derivative of the loss with respect to the input. In the case of a linear layer, in addition to the derivative, it accumulates the gradient of the loss with respect to its parameters. 
    
\paragraph{param(self)}
Returns a tuple of tenors containing the parameters of the model and their corresponding derivatives. 
    
\paragraph{reset(self)}
Resets the gradient accumulation of the linear layer to 0
    
\paragraph{update(self,eta)}
Updates the parameters of the module according to the gradient. The size of the step is given by the parameter eta defined by the user. 






\end{document}

%
%
%
%
%
%
%
%
%
%
%
%
%
%
%
%
%
%
%
%
%
%
%
%
%
%
%
